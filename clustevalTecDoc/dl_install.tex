

	\section{Download \& Installation}\label{sec_inst}
		There are basically two options how to get a working installation of \clusteval. The easy way is to download our VirtualBox Image and start the virtual machine. The image contains both, the website and the backend server- and client executables. This procedure is explained in \ref{subsec_vbox_image}. The second option is to install the framework manually step-by-step. The requirements of the backend and its installation are explained in \ref{backend_requirements} and \ref{install_backend}. respectively. The requirements of the frontend and its installation are detailed in \ref{frontend_requirements} and \ref{install_frontend} .
		\subsection{VirtualBox Image}\label{subsec_vbox_image}
		The first option how to get a running \clusteval is to download the VirtualBox from our server, start the machine and have everything installed right from the start. You can then access the frontend website in the webbrowser, start the backend server and client and produce your first clustering results. You do not have to install any unmet requirements or do any configurations. Here is how:
			\begin{enumerate}
				\item Install VirtualBox
				\item Download the VirtualBox disk image from our \href{\vboxImage}{server} and start the machine
				\item The backend server can be found under 
				
				\url{\vboxServer}
				\item The backend client is located under 
				
				\url{\vboxClient}
				\item To access the frontend website, open up the preinstalled webbrowser and open the address 
				
				\url{\vboxFrontend}
			\end{enumerate}
	\subsection{Manual Installation}
	To get \clusteval running on your own server, you have to install it manually. The following sections will show you, what requirements have to be met for each component of \clusteval (backend server, client and frontend website, database) and how these components can be installed.
	
		 \subsection{Backend Requirements}\label{backend_requirements}
	The backend (including server and client component) uses several java libraries that already ship within the corresponding binaries (jar-files). They have to be available within the class path of \clusteval when it is started. 
	
	Additionally many functions of the backend server require a working R installation (version $\geq$ 2.15) together with an installed Rserve library within R. In principle the clustering processes of \clusteval can be started \textbf{without} a running \textbf{Rserve}, but you will not be able to use any functions depending on R. Throughout this manual, functionalities which require R will be labeled with $^{(Rserve)}$.
	
	\subsubsection{Java $\geq$ 1.6}
	\clusteval is completely written in Java and requires a JRE $\geq$ 1.6 on the target system.
	\subsubsection{Java libraries}
	The following Java libraries have to be accessible within the classpath when \clusteval is started. The versions in brackets are the versions, that are by default shipped with the framework. \clusteval has been tested together with these versions, so we cannot guarantee functionality, when different versions are used.
	
	\paragraph{Apache Commons CLI (1.2)} is a library for parsing command line arguments and printing CLI usage
	\paragraph{Apache Commons Configuration (1.8)} provides classes and parsers for configuration files
	\paragraph{Apache Commons IO (2.3)} provides file operations like copying and moving
	\paragraph{Apache Commons Lang (2.6 \& 3.1)} Base library of apache commons
	\paragraph{Apache Commons Logging (1.1.1)} is a library used by other apache commons libraries for logging
	\paragraph{JAnsi (1.9)} provides ANSI codes on the console output
	\paragraph{JLine (2.8 Snapshot)} a library for command line tab completion
	\paragraph{JUnit} provides classes for testing and validation
	\paragraph{JUnit Addons (1.4)} an extension library of JUnit
	\paragraph{log4j (1.2.16)} an logging API for Java
	\paragraph{Logback (1.0.9)} a logging library based on log4j
	\paragraph{Parallel Colt (0.9.4)} a library for fast and memory efficient methods and data structures is used by \clusteval to store data matrices of datasets
	\paragraph{MySQL Connector Java (5.1.21)} enables Java to communicate with MySQL databases
	\paragraph{Rserve (1.7)} libraries providing possibilities to use the R framework from within Java classes
	\paragraph{TransClust (1.0)} a clustering method, is used as a library to parse and convert FASTA and BLAST files
	\paragraph{Wiutils (1.0)} a java library providing several general-purpose functionalities and classes, mostly inspired by R convenience functions
	\subsubsection{[R installation ($\geq$ 2.15)]}	
	R-depending components introduced into \clusteval can make use of functions of arbitrary R packages. R is in principle optional, but then those functions will not be available. Only subclasses of the following classes are by design intended to make use of R:
	\begin{itemize}
		\item \highlight{de.clusteval.cluster.quality.ClusteringQualityMeasure}
		\item \highlight{de.clusteval.data.distance.DistanceMeasure}
		\item \highlight{de.clusteval.data.generator.DataSetGenerator}
		\item \highlight{de.clusteval.program.r.RProgram}
		\item \highlight{de.clusteval.cluster.paramOptimization.ParameterOptimizationMethod}
		\item \highlight{de.clusteval.data.statistics.DataStatistic}
		\item \highlight{de.clusteval.run.statistics.RunStatistic}
		\item \highlight{de.clusteval.run.statistics.RunDataStatistic}
	\end{itemize}
	
	Subclasses of these classes need to tell the framework which R packages they require. The corresponding class is only loaded, if these R packages are available.
	
	R is available for Unix, MacOS and Windows systems and can be downloaded from \url{http://cran.r-project.org/}. The Rserve package requires a R version $\geq$ 2.15. Some systems (e.g. Debian 6) do not have these R versions in their default repositories. In these cases you can download the binaries from the official website and install them manually. If you are on a Debian system, you can add the CRAN repositories to your aptitude package system and install it via console. For Debian 6 this can be achieved by doing the following steps:
	
	\begin{enumerate}
		\item Append the aptitude repository to your sources.list
		
		e.g. 'deb http://cran.cnr.berkeley.edu/bin/linux/debian squeeze-cran/'
		\item Update your local aptitude package information
		
		\codei{apt-get update}
		\begin{itemize}
			\item If the update fails with an error, that you have not authenticated the new server, you have to import the public key of the server with the following command
			
			\codei{gpg --keyserver subkeys.pgp.net --recv-keys <serverId>}
			
			In case of the above example CRAN repository server:
			
			\texttt{<serverId> = 06F90DE5381BA480}
			
			\item Then you will have to add the imported public key of the server to the authorized server keys of aptitude 
			
			\codei{gpg -a --export <serverId> | sudo apt-key add -}
			\item Rerun \codei{apt get update}
		\end{itemize}
		\item Install the R framework by executing
		
		\codei{apt-get install r-base}
	\end{enumerate}
	
	If you are not familiar with R but are interested in some more information about it, you can also find good documentations on the aforementioned website.
	\paragraph{Rserve package}
	Rserve acts as a distribution server to make R available via TCP/IP. Different interfaces are provided for Rserve, such that R can be used from within other programs written in different programming languages, for example Java. This functionality is exploited by \clusteval in that statistical calculations are outsourced to existing R implementations instead of implementing them from scratch in Java.
	
	This is used in several components of \clusteval, which are listed below. In principle the clustering processes of \clusteval can be started without R or a running Rserve instance, but you will not be able to use any functions depending on R.
	
	\subparagraph{Installation} To install the Rserve package in your R installation, open your R console. On Unix systems you type R in your console and a R terminal will open up:
	
	\codei{R}
	
	Now you can install the Rserve package using the following command:
	
	\codei{install.packages("Rserve")}
	
	After you have installed the Rserve package in R, you have to start an instance of Rserve, which will listen on a specific port for incoming calls.
	
	\codei{Rserve()}
	
	 In our case these calls will origin from the backend server, which connects to Rserve. This happens without further intervention by the user.
	
	\paragraph{R libraries}
	Several components that ship with \clusteval require certain R packages. If you want to use the depending components, you will have to install these packages using the \codei{install.packages("<packageName>")} command.
	
	
	\subparagraph{"cluster"} is required by
	\begin{itemize}		
		\item \highlight{GapStatisticParameterOptimizationMethod}
		\item \highlight{SilhouetteValueRClusteringQualityMeasure}
\end{itemize}

	\subparagraph{"fields"} is required by
	\begin{itemize}		
		\item \highlight{HopkinsDataStatistic}
\end{itemize}
	
	\subparagraph{"Hmisc"} is required by
	\begin{itemize}
		\item \highlight{HoeffdingDRDistanceMeasure}
		\item \highlight{PearsonCorrelationRDistanceMeasure}
		\item \highlight{SpearmanCorrelationRDistanceMeasure}
	\end{itemize}
	
	\subparagraph{"igraph"} is required by
	\begin{itemize}		
		\item \highlight{GraphMinCutRDataStatistic}
		\item \highlight{GraphDiversityAverageRDataStatistic}
		\item \highlight{GraphDensityRDataStatistic}
		\item \highlight{GraphCohesionRDataStatistic}
		\item \highlight{GraphAdhesionRDataStatistic}
		\item \highlight{ClusteringCoefficientRDataStatistic}
		\item \highlight{ClusteringCoefficientRDataStatistic}
\end{itemize}
	
	\subparagraph{"kernlab"} is required by 
	\begin{itemize}
		\item \highlight{SpectralClusteringRProgram}
	\end{itemize}
	
	\subparagraph{"lattice"} is required by 
	\begin{itemize}
		\item \highlight{CooccurrenceBestRunStatistic}
		\item \highlight{CooccurrenceRunStatistic}
	\end{itemize}
	
	\subparagraph{"MASS"} is required by 
	\begin{itemize}
		\item \highlight{LinearModelRidgeRunDataStatistic}
	\end{itemize}

	\subparagraph{"mlbench"} is required by
	\begin{itemize}
		\item \highlight{CassiniDataSetGenerator}
		\item \highlight{CircleDataSetGenerator}
		\item \highlight{CuboidDataSetGenerator}
		\item \highlight{Gaussian2DDataSetGenerator}
		\item \highlight{HyperCubeCornersDataSetGenerator}
		\item \highlight{RingnormDataSetGenerator}
		\item \highlight{SimplexCornersDataSetGenerator}
		\item \highlight{SpiralsDataSetGenerator}
	\end{itemize}
	
	\subparagraph{"stats"} is required by 
	\begin{itemize}
		\item \highlight{HierarchicalClusteringRProgram}
		\item \highlight{KMeansClusteringRProgram}
	\end{itemize}
	
		 \subsection{Backend Installation}\label{install_backend}
		 In the following we provide a step by step installation manual for Debian 6 servers. We expect the aforementioned requirements to be installed and fulfilled including a Java and R installation.
		 	\begin{itemize}
				\item Download server and client executables from \href{\urlserver}{here} and \href{\urlclient}{here}.
				\item Execute backend server by typing
				
				\codei{java -jar clustevalServer.jar}
				
				This will start the backend server with default options, listening on port 1099 for clients and using the folder repository in the current directory as repository root. The repository root is recreated, if it does not exist. If it does exist, everything will be loaded and parsed only from this directory (see \ref{subsec_repository} for more information on Repositories).
				
				For help on available command line parameters you can check out \ref{backend_server_cli} or invoke the backend server with the help parameter:
				
				\codei{java -jar clustevalServer.jar -help}
		 	\end{itemize}
		 \subsection{Frontend Requirements}\label{frontend_requirements}
		 The frontend requires a MySQL installation together with a database, that has all necessary tables in it. The backend server has to be configured to use this database and fill it with all relevant data contained in the backend repository. The website itself is based on Ruby on Rails and therefore requires an installation of Ruby on Rails. 
		 
		 Throughout development and for testing as well as validation we used MySQL 5.1 and RoR 1.9.1. For other versions we cannot guarantee the website behaving correctly.
		 
		 \subsubsection{MySQL (5.1)}
		 If the \clusteval frontend website is to be used, the MySQL database is required to store and retrieve all the data visualized on the website. This data in turn is inserted into the database by the backend server. To install MySQL under Debian 6, you can type
		
		\codei{sudo apt-get install mysql-server}
		
		During the installation procedure this will ask you for a root password, and the credentials of a non-root user. We assume these to be \highlight{clustEvalRead}
		
		 We can only ensure full compatibility of \clusteval with the version of MySQL that is indicated in brackets.
		 
		 \paragraph{Database \& User setup}
		 
		 The website will require a database and a user with read-rights for that database. It is \textbf{highly discouraged} to use the \textbf{root user} or any other user with write-rights for this purpose. In case the website contains potential security holes, this can lead to misuse and damage to the MySQL database. If the website uses a user with read-rights, this cannot happen. To create a mysql database, open up a MySQL terminal by typing
		 
		 \codei{mysql -u <YOUR\_USERNAME> -p}
		 
		 this will ask you for your password. To create a database, perform the following command
		 
		 \code{mysql> CREATE DATABASE <DB\_NAME>;}
		 
		 Now we create the user which will be used by the website to access the database.
		 
		 \code{mysql> CREATE USER 'clustEvalRead'@'localhost' IDENTIFIED BY '<YOUR\_PASSWORD>';}
		 
		 We grant our user the right, to select from the database and to show views.
		 
		 \code{mysql> GRANT SELECT ON <DB\_NAME>.* TO clustEvalRead;}
		 
		 \code{mysql> GRANT SHOW VIEW ON <DB\_NAME>.* TO clustEvalRead;}
		 
		 \subsubsection{Ruby on Rails (1.9.1)}
		 The frontend website is developed in Ruby on Rails. Therefore the website requires several components on the target machine:
		 \begin{itemize}
		 	\item Ruby, the programming language
		 	\item a webserver compatible to Ruby on Rails
		 \end{itemize}
		 In this guide we will be using Apache with the passenger extension as webserver and guide you through the installation process on a Debian 6 server. The passenger extension adds Ruby on Rails support to Apache webservers.
		 
		 We can only ensure full compatibility of \clusteval with the version in brackets. In the following steps we will denote the version of ruby as \texttt{\highlight{<RUBY\_VERSION>}}.
		 
		 \begin{enumerate}
		 	\item To install Ruby, use the command:
		 	
		 	\codei{apt-get install ruby<RUBY\_VERSION>-full}

		 	This will install ruby, all of its dependencies and developmental libraries of the chosen ruby version. 
		 	We create some symlinks such that the ruby executables are available without providing the version number. If you do not create these symlinks, this can cause problems during the installation process.
		 	
		 	\codei{ln -s /usr/bin/ruby<RUBY\_VERSION> /usr/bin/ruby}
		 	
			\codei{ln -s /usr/bin/gem<RUBY\_VERSION> /usr/bin/gem}
			
			\codei{ln -s /var/lib/gems/<RUBY\_VERSION>/bin/bundle /usr/bin/bundle}
			
			You can verify, that the creation was successful, by using the \highlight{which} command and the \highlight{-v} command line switch of the corresponding commands.
			
			We install the bundler gem, which will be needed later on to install gem dependencies of our website.
			
			\codei{gem install bundler}
			
		 	\item To install Apache:
		 	
		 	\codei{apt-get install apache2}
		 	
		 	This will install the apache webserver with a default setup. We will modify this later on by adding a virtual host.
		 	
		 	\item Now we can install the passenger ruby gem, which will provide us with a script that installs the apache passenger module.
		 	
		 	\codei{gem install passenger}
		 	
			We assume the passenger gem has been installed to
			
		 	\highlight{\texttt{/var/lib/gems/<RUBY\_VERSION>/gems/passenger-<PASSENGER\_VERSION>/}}
		 	
		 	such that in the subfolder \highlight{bin} we can find and execute the script
		 	
			\codei{passenger-install-apache2-module}
			
			The last command will give you some hints, about missing packages. For every missing package the install command is provided. Please follow these hints and install the missing packages. Also the passenger installer will tell you, how you have to modify your \highlight{/etc/apache2/apache2.conf} in order for your apache webserver to load the passenger module.
		 \end{enumerate}
		 
		 \subsection{Frontend}\label{install_frontend}
		 At this point Ruby and Apache together with the passenger module are assumed to be installed. Now you can install the frontend website, by downloading a zip-file from our server containing the \clusteval website. After extracting the archive and placing the website at your favored location, you will use some ruby commands, to install all remaining package dependencies (gems) required by the website.
		 \begin{enumerate}
		 	\item Download and extract the latest version of the frontend from our \href{\urlfrontend}{website}. Place the extracted contents in your favored directory. We will denote this path by \texttt{<WEBSITE\_ROOT>} in the following steps.
		 	\item In your terminal navigate to \texttt{<WEBSITE\_ROOT>} and execute the command 
		 	
			\codei{bundle install --no-deployment}

			\codei{bundle install --deployment}
		 	
		 	This will install all gems required by the website. For this process an internet connection is required.
		 	\item Create an Apache virtual host by creating the file '/etc/apache2/sites-available/clusteval' with the following content and replace the placeholders \texttt{<YOUR\_DOMAIN>} and \texttt{<WEBSITE\_ROOT>} with your values.
	\begin{verbatim}
<VirtualHost *:80>
      ServerName <YOUR_DOMAIN>
      DocumentRoot <WEBSITE_ROOT>/public
      <Directory <WEBSITE_ROOT>/public>
         # This relaxes Apache security settings.
         AllowOverride all
         # MultiViews must be turned off.
         Options -MultiViews
      </Directory>
</VirtualHost>
	\end{verbatim}
		 	\item You can use the apache2 virtual host configuration command to enable your new clusteval virtual host
		 	
		 	\codei{a2ensite clusteval}
		 	
		 	Afterwards you can restart your Apache Webserver, for the new settings to take effect.
		 	
		 	\codei{apache2ctl restart}
		 	
		 	\item Now we tell the website which database to use by adapting the database name and user details in \highlight{$<$WEBSITE\_ROOT$>$/config/database.yml}.
		 	
		 	\item To initialize the database with a default state, perform
		 	
		 	\codei{rake db:reset}
		 \end{enumerate}