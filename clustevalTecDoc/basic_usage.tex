
		 
	\section{Basic Usage}\label{sec_usage}
	After you downloaded and installed \clusteval, you need to know how to use it. The backend server and client are distributed as two executables which can be simply started with a single command in the command line. Both can be invoked with different command line parameters, which will be explained in the next sections.
	
	For the following scenarios we assume, that the website is ready to use.
	\subsection{Backend server}\label{usage_backend_server}
	The backend server is available as a jar-executable \highlight{\backendserver}. It can be executed from the command line by performing
	
	\code{java -jar \backendserver}
	
	There are several command line options available, which are listed in \ref{backend_server_cli}. This command executes the server, which will parse the contents of the repository and wait for commands from clients. Log messages of the server will be printed to the console and to a log file in the invocation directory.
	\subsubsection{Command line}\label{backend_server_cli}
	The backend server provides several command line options:
	\begin{description}
	\item[-absRepoPath $<$absRepositoryPath$>$] is the absolute path to the repository which should be used by the backend server to store results and parse any data from.
	\item[-help] prints help and usage information.
	\item[-logLevel $<$level$>$] defines the log level of the server. This determines, how verbose the server will be during its execution. Default is 3=INFO. All available logging levels are: 0=ALL,
                                    1=TRACE, 2=DEBUG, 3=INFO, 4=WARN,
                                    5=ERROR, 6=OFF
	\item[-serverPort $<$port$>$] is the port the backend server should listen for clients
	\end{description}
	

		
		\subsection{Backend client}\label{usage_backend_client}
		The backend client is also available as a jar-executable \highlight{\backendclient} and can be executed from the command line by performing
		
		\code{java -jar \backendclient}
		
		Invoking the client without any parameters besides "clientId", "hostport", "hostip", "logLevel", "help" or "waitForRuns" will open up a tab-completed shell, in which commands can be committed. If any other parameters are specified, the client will directly perform them and terminate immediately afterwards. The available command line parameters are shown in \ref{backend_client_cli}.
		\subsubsection{Command line}\label{backend_client_cli}
	Supported commands of the backend client are:
	\begin{description}
	\item[generateDataSet $<$generatorName$>$]: This command can be used to generate synthetic datasets using the specified generator. 
	\item[-getDataSets]: This tells the client to get and print the available datasets from the server.
\item[-getPrograms]: This tells the client to get and print the available programs from the server.
\item[-getQueue]: Gets the enqueued runs and run resumes from the backend server.
\item[-getRuns]: This tells the client to get and print the available runs from the server.
\item[-getRunResumes]: This tells the client to get and print all the run result directories contained in the repository of this server. Those run result directories can be resumed, if they were terminated before.
\item[-getRunResults]: This tells the client to get and print all the run result directories contained in the repository of this server, that contain a clusters subfolder and at least one *.complete file containing results (can be slow if many run result folders are present).
\item[-getRunStatus]: This tells the client to get the status and percentage (if) of a certain run. 
\item[-help] prints this help and usage information
\item[-hostip $<$ip$>$] specifies the ip address of the backend server to connect to. Defaults to the localhost.
\item[-hostport $<$port$>$] specifies the port number of the backend server to connect to. Defaults to port 1099.
\item[-logLevel $<$level$>$] defines the log level of the client. This determines, how verbose the client should be during its execution. Available logging levels are: 0=ALL, 1=TRACE, 2=DEBUG, 3=INFO, 4=WARN, 5=ERROR, 6=OFF
\item[-performRun $<$runName$>$]: This tells the client to perform a run with a certain name.
\item[-resumeRun $<$runResultIdentifier$>$]: This tells the client to resume a run previously performed identified by its run result identifier.
\item[-shutdown]: Orders the backend server to terminate gracefully.
\item[-terminateRun $<$runName$>$]: This tells the client to terminate the execution of a run with a certain name.
shutdown: This tells the client to shutdown the framework.
\item[-waitForRuns]: This option can be used together with getRunStatus, in order to cause the client to wait until the run has finished its execution.
	\end{description}
	\subsection{Creating and Executing a Parameter Optimization Run}
	\begin{description}		
	\item[Goal:] You want to create a new parameter optimization run \highlight{newParamOptRun}, that applies two clustering methods
	\begin{itemize}
		\item \highlight{awesomeCluster} and 	
		\item \highlight{groupTest}
\end{itemize}
to two datasets together with their goldstandards
	\begin{itemize}
		\item \highlight{myGeneExpressionData\_3.txt} (\highlight{geneExpr\_3\_goldstandard.txt}) and
		\item \highlight{methAbs.txt} (\highlight{methylation\_gs.txt}).
	\end{itemize}
The run should assess three clustering quality measures for every clustering
	\begin{itemize}
		\item \highlight{SilhouetteValueRClusteringQualityMeasure}
		\item \highlight{TransClustF2ClusteringQualityMeasure}
		\item \highlight{SensitivityClusteringQualityMeasure}
	\end{itemize}
	
	You want to optimize the \highlight{TransClustF2ClusteringQualityMeasure} and use the 
	\highlight{LayeredDivisiveParameterOptimizationMethod} to determine the parameter sets that should be evaluated during the optimization process.
	 \item[Prerequisites:] You downloaded \clusteval backend server and client, have put the executables into the desired directory. We assume this directory to be \highlight{$<$clustEvalDirectory$>$}, such that the executables are located at
	 
	 \begin{itemize}
		\item \highlight{$<$clustEvalDirectory$>$/\backendserver} and 
		\item \highlight{$<$clustEvalDirectory$>$/\backendclient}
	 \end{itemize} When the backend server is started without an absolute repository path, it will create (if it does not exist yet) a new repository in the folder \highlight{$<$clustEvalDirectory$>$/repository}. We assume the repository path to be \highlight{$<$REPOSITORY\_ROOT$>$}.
	The repository is empty, such that we need to create all configuration files and put all input files into the corresponding folders of the repository.
	
	\item[Input files:] First you need to put the two dataset and goldstandard files into the repository. For this purpose you have to think of a meaningful foldername containing the dataset/goldstandard. In this case we choose \textit{expression} for the first and \textit{methylation} for the second dataset. Thus we copy the files into the corresponding folders:
	\begin{itemize}
		\item \highlight{myGeneExpressionData\_3.txt} to
		
		 \repodatasets/expression/myGeneExpressionData\_3.txt
		 \item \highlight{geneExpr\_3\_goldstandard.txt} to
		
		 \repogoldstandards/expression/geneExpr\_3\_goldstandard.txt
	\end{itemize}
	and
	\begin{itemize}
		\item \highlight{methAbs.txt} to
		
		 \repodatasets/methylation/methAbs.txt
		 \item \highlight{methylation\_gs.txt} to
		
		 \repogoldstandards/methylation/methylation\_gs.txt
	\end{itemize}
	
	You need to make sure, that the dataset files have a valid header in them. In our example the dataset \highlight{myGeneExpressionData\_3.txt} has the format \highlight{RowSimDataSetFormat} (version 1) and \highlight{methAbs.txt} has the format \highlight{MatrixDataSetFormat} (version 1). As the names indicate, the first dataset has the type \highlight{GeneExpressionDataSetType} and the second one has \highlight{MethylationDataSetType}. Since the latter is not available in \clusteval by default, you need to add it yourself (see \ref{subsec_extend_datasettypes}).
	
	Now the headers of your two dataset files look as follows:
	
	\begin{lstlisting}
// dataSetFormat = RowSimDataSetFormat
// dataSetType = GeneExpressionDataSetType
// dataSetFormatVersion = 1
	\end{lstlisting}
	
and

	\begin{lstlisting}
// dataSetFormat = MatrixDataSetFormat
// dataSetType = MethylationDataSetType
// dataSetFormatVersion = 1
	\end{lstlisting}
	
	These headers have to be the first lines in your dataset.
	
	\item[Dataset, Goldstandard \& Data Configurations:] Now you need to create configuration files for the newly inserted data files.
	
	We create two dataset configuration files under 
	\begin{itemize}
		\item \highlight{\repodatasetconfigs/geneExpr.dsconfig}
		\begin{lstlisting}
datasetName = expression
datasetFile = geneExpr_3_goldstandard.txt
		\end{lstlisting}
		\item \highlight{\repodatasetconfigs/methyl.dsconfig}
		\begin{lstlisting}
datasetName = methylation
datasetFile = methylation_gs.txt
		\end{lstlisting}
	\end{itemize}
	
	We create two goldstandard configuration files under 
	\begin{itemize}
		\item \highlight{\repogoldstandardconfigs/geneExpr.gsconfig}
		\begin{lstlisting}
goldstandardName = expression
goldstandardFile = Zachary_karate_club_gold_standard.txt
		\end{lstlisting}
		\item \highlight{\repogoldstandardconfigs/methyl.gsconfig}
		\begin{lstlisting}
datasetName = methylation
datasetFile = methAbs.txt
		\end{lstlisting}
	\end{itemize}
	
	And finally we create two data configurations which combine the dataset and goldstandard configurations:
	\begin{itemize}
		\item \highlight{\repodataconfigs/geneExpr.dataconfig}
		\begin{lstlisting}
datasetConfig = geneExpr
goldstandardConfig = geneExpr
		\end{lstlisting}
		\item \highlight{\repodataconfigs/methyl.dataconfig}
		\begin{lstlisting}
datasetConfig = methyl
goldstandardConfig = methyl
		\end{lstlisting}
	\end{itemize}
	
	\item[Clustering Methods:] To add your two clustering methods we again need to think of a foldername and put the executables into the repository:
	\begin{itemize}
		\item \highlight{\repoprograms/awesomeCluster/awesomeCluster}
		\item \highlight{\repoprograms/groupTest/groupTest}
	\end{itemize}
	
	\item[Program Configurations:] Now you need to create configuration files for each of these clustering methods and put them into the folder \highlight{\repoprogramconfigs}. They might be located at
	\begin{itemize}
		\item \highlight{\repoprogramconfigs/awesomeCluster.config}
		\item \highlight{\repoprogramconfigs/groupTest.config}
	\end{itemize}
	and the contents might look like this:
	
	\begin{lstlisting}
program = awesomeCluster/awesomeCluster
parameters = a,w
optimizationParameters = a
compatibleDataSetFormats = RowSimDataSetFormat
outputFormat = AwesomeRunResultFormat
alias = Awesome Cluster

[invocationFormat]
invocationFormat = %e% --input %i% --output %o% -a %a% -b %b% -c %c%

[a]
desc = The a parameter
type = 2
def = 1002
minValue = 900
maxValue = 2000

[w]
desc = The w parameter
type = 2
def = 0.001
minValue = 0.0
maxValue = 0.02
	\end{lstlisting}
	and
	
	\begin{lstlisting}
program = groupTest/groupTest
parameters = g
optimizationParameters = g
compatibleDataSetFormats = RowSimDataSetFormat,SimMatrixDataSetFormat
outputFormat = GroupRunResultFormat
alias = Group-Test Clustering

[invocationFormat]
invocationFormat = %e% -g %g% %i% %o% -silent

[g]
desc = The grouping coefficient
type = 2
def = 1
minValue = 0
maxValue = 15.3
	\end{lstlisting}
	
	In this case we have to follow the explanations under \ref{subsec_extend_runresultformats} to add the two runresult formats of the clusterin methods to our framework.
	
	\item[Run:] Now we create the parameter optimization run file under
	
	\highlight{\reporuns/myNewRun.run}
		
	 with the details specified above:
\begin{lstlisting}
programConfig = awesomeCluster,groupTest
dataConfig = methyl,geneExpr
qualityMeasures = TransClustF2ClusteringQualityMeasure,SilhouetteValueRClusteringQualityMeasure,SensitivityClusteringQualityMeasure
mode = parameter_optimization
optimizationMethod = LayeredDivisiveParameterOptimizationMethod
optimizationCriterion = TransClustF2ClusteringQualityMeasure
optimizationIterations = 100

[awesomeCluster]
optimizationParameters = a

[groupTest]
optimizationParameters = g
	\end{lstlisting}
	\item[Executing:] To execute our new run, we start the backend server
	
	\codei{java -jar $<$clustEvalDirectory$>$/\backendserver}
	
	we wait until everything contained in the repository is parsed. Then we start the client
	
	\codei{java -jar $<$clustEvalDirectory$>$/\backendclient}
	
	This will open up a shell and we can start our new run by typing
	
	\code{localhost:1099> performRun myNewRun}
\end{description}	